\begin{figure}[t!]
% \begin{minipage}[b]{0.49\linewidth}
\vspace{-0.5cm}
\begin{tikzpicture}[scale=0.9, >= stealth, shorten > = 1pt, auto, semithick]
        \tikzstyle{every state}=[draw = black, thick, fill = white, minimum size = 4mm]
        \def\C{0.7}
        \node[state] (A) at (0,0) {$A$};
        \node[state] (C) at (\C*2, -1*\C) {$C$};
        \node[state] (B) at (2*\C, 1*\C) {$B$};
        \node[state] (D) at (4*\C, 1*\C) {$D$};
        \node[state, draw = none] (X) at (0*\C, 2*\C) {};
        \node[state, draw = none] (Y) at (4*\C,-1*\C) {};
        \node[state, draw = none] (Z) at (0*\C,-1*\C) {};

        \path[-{Latex[width=3mm]}] (X) edge node {1} (A);
        \node (t) at ($(A)!0.5!(B)$) {2};
        \path[-{Latex[width=3mm]}] (B) edge node {3} (D);
        \path[-{Latex[width=3mm]}] (D) edge node {4} (Y);
        \path[Latex-{Latex[width=3mm]}] (D) edge node [above]{5} (C);
        \path[-{Latex[width=3mm]}] (C) edge node[above] {6} (Z); 
                
        \connectThree[@edge 1=-{Latex[width=3mm]}, @edge 2=-{Latex[width=3mm]}]{C}{B}{A}
        
        \def\S{6}
        \def\W{-0.7}
        
        \node[state] (BB) at (0*\C+\S, 0*\C+\W) {$B'$};
        \node[state] (AA) at (0*\C+\S, 2*\C+\W) {$A'$};
        \node[state, draw = none] (UU) at (-2*\C+\S, 0*\C+\W) {};
        \node[state, draw = none] (VV) at (2*\C+\S, 1*\C+\W) {};
        \node[state, draw = none] (WW) at (2*\C+\S, 1*\C+\W) {};
        \node[state, draw = none] (XX) at (-2*\C+\S, 1*\C+\W) {};
        \node[state, draw = none] (YY) at (-2*\C+\S,2*\C+\W) {};
        \node[state, draw = none] (ZZ) at (2*\C+\S,2*\C+\W) {};

        \path[Latex-{Latex[width=3mm]}] (YY) edge node {1} (AA);
        \path[Latex-{Latex[width=3mm]}] (XX) edge node {2} (AA);
        \path[Latex-{Latex[width=3mm]}] (ZZ) edge node[above] {3} (AA);
        \path[{Latex[width=3mm]}-Latex] (AA) edge node[left] {4} (BB);
        \path[Latex-{Latex[width=3mm]}] (UU) edge node {5} (BB); 
        \node (t) at ($(BB)!0.5!(VV)$) {6};
                
        \connectThree[@edge 1=-{Latex[width=3mm]}, @edge 2=-{Latex[width=3mm]}, @edge 3=-Latex]{AA}{BB}{VV}
        
        
    \end{tikzpicture}
\footnotesize{  
\begin{minipage}[b]{0.49\linewidth}
\vspace{-0.3cm}
\[
  \begin{bmatrix}
    1 & -1 & 0 & 0 & 0 & 0 \\
    0 & 1 & -1 & 0 & 0 & 0 \\
    0 & 1 & 0 & 0 & 1 & -1 \\
    0 & 0 & 1 & -1 & -1 & 0\\
  \end{bmatrix}
\]

\centering
Stoichiometry matrix\\
\vspace{0.3cm}
{[}No  No  No  No  Yes  No{]} \\
Reversibility vector 
\end{minipage}
\begin{minipage}[b]{0.49\linewidth} 
    \[
  \begin{bmatrix}
    1 & 1 & 1 & 1 & 0 & 1 \\
    0 & 0 & 0 & -1 & 1 & 1 
  \end{bmatrix}
\]
\centering
Nullspace matrix\\
\vspace{0.3cm}
{[}Yes  Yes Yes  Yes Yes  Yes{]} \\
Reversibility vector
\end{minipage}
}

\centering    
\begin{tikzpicture}
    \node at (-1.1, 0.75) {Some cut };
    \node at (-1.1, 0.5) {sets for };
    \node at (-1.1, 0.25) {reaction $1$};
    \node at (4.0, 0.75) {Some FMs};
    \node at (4.0, 0.5) {in the dual};
    \node at (4.0, 0.25) {network};
    \node at (2.0, 0)   {$[1,0,0,0,1,-1]$};    
    \node at (2.0, 0.5) {$[1,1,-2,0,0,0]$};
    \node at (2.0, 1)   {$[1,0,0,\frac{-1}{2},0,\frac{-1}{2}]$};    
    \node at (0, 0) {$\{5,6\}$};
    \node at (0, 0.5) {$\{3\}$};    
    \node at (0, 1) {$\{4,6\}$};
    \draw[Stealth-] (0.4,0) -- (1,0);
    \draw[Stealth-] (0.4,.5) -- (1,.5);
    \draw[Stealth-] (0.4,1) -- (0.8,1);
    % \node at (2.4, 0.2) {\footnotesize{$\mathcal{I}$-Coordinated support}};
    % \node at (2.4, 0.7) {\footnotesize{$\mathcal{I}$-Coordinated support}};
    % \node at (2.4, 1.2) {\footnotesize{$\mathcal{I}$-Coordinated support}};
    \draw[decoration={brace,raise=2pt},decorate]
        (-0.3,0) -- node[left=3pt] {} (-0.3,1);
    \draw[decoration={brace,raise=2pt},decorate]
        (3.1,1) -- node[right=3pt] {} (3.1,0);
\end{tikzpicture}
    % \caption{Reaction 6, 7, and 8 are a (minimal) cut-set for reaction 3. But the splitting claim won't hold here.}
    % \label{fig:counter_example}
    % \tamon{Old caption moved to text, "row-space network": suitable term?}
    \caption{Example of a metabolic network with its associated dual network. Some of its FMs involving target reaction $1$ are shown; their $\mathcal{I}-$coordinated supports result in cut sets for it in the original network.}
    
    \label{fig:example}
    % \tamon{Old caption moved to text, "row-space network": suitable term?}
    \vspace{-0.5cm}
\end{figure}